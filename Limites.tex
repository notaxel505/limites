\documentclass[a4paper,10pt]{article}
%\documentclass[a4paper,10pt]{scrartcl}

\usepackage[utf8]{inputenc}
\usepackage[margin=2cm]{geometry}
\usepackage{amsmath}
\usepackage{cancel}

\title{Límites}
\author{Axel Ulices Alcantar Gutiérrez}
\date{13 Sep 2025}

\pdfinfo{%
  /Title    (Límites)
  /Author   (Axel Ulices Alcantar Gutiérrez)
  /Creator  ()
  /Producer ()
  /Subject  ()
  /Keywords (Límites Matemáticas)
}

\begin{document}
\maketitle

\section{Límite con indeterminación 0/0}

\begin{enumerate}
	\item Límite a resolver
	      $$\lim_{x\to2}\frac{x^{2} - 4}{4x - 8}$$
	      Cuando se resuelve un límite lo primero que se tiene que hacer es sustituir el valor de la variable.

	\item Sustituir valores
	      $$\frac{(2)^{2} - 4}{4(2) - 8} = \frac{4 - 4}{8 - 8} = \frac{0}{0}$$

	      $\frac{0}{0}$ es una indeterminación, para resolver esto podemos factorizar. Cuando la ecuación tenga una forma $(a^2 - b^2)$ se puede factorizar de la manera $(a + b)(a - b)$. A esto se le llama \textbf{diferencia de cuadrados}.

	\item Factorizar numerador
	      $$\frac{x^{2} - 4}{4x - 8} = \frac{(x + 2)(x - 2)}{4x - 8}$$

	      Como puedes ver, en este momento todavía no podemos hacer sustitución directa, pero para resolver esto podemos factorizar el denominador diviendo los términos entre 4.

	\item Factorizar denominador
	      $$\frac{(x + 2)(x - 2)}{4x - 8} = \frac{(x + 2)(x - 2)}{4(x - 2)}$$

	      La ecuación ahora sí puede cambiar cancelando $(x - 2)$ tanto del numerador como del denominador.

	\item Cancelando términos
	      $$\frac{(x + 2)(x - 2)}{4(x - 2)} = \frac{(x + 2)\cancel{(x - 2)}}{4\cancel{(x - 2)}} = \frac{x + 2}{4}$$

	      La ecuación ha tomado un nuevo rumbo, es hora de sustituir los valores de nuevo.

	\item Sustituir valores de nuevo
	      $$\frac{2 + 2}{4} = \frac{4}{4} = 1$$

	      Ahora no hay indeterminación, lo que significa que hemos resuelto el límite.

	\item Resultado
	      $$\boxed{\lim_{x\to2}\frac{x^{2} - 4}{4x - 8} = 1}$$

\end{enumerate}

\section{Límite con indeterminación 0/0}
\begin{enumerate}
	\item Límite a resolver
	      $$\lim_{x\to4}\frac{x^{2} - 4x}{x^{2} - 16}$$

	\item Sustituir valores
	      $$\frac{(4)^{2} - 4(4)}{(4)^{2} - 16} = \frac{16 - 16}{16 - 16} = \frac{0}{0}$$

	      Es una indeterminación, por lo que tendremos que factorizar. En el numerador podemos factorizar por medio del factor común.

	\item Factorizar numerador
	      $$\frac{x^{2} - 4x}{x^{2} - 16} = \frac{x(x - 4)}{x^{2} - 16}$$

	      Ahora tendremos que factorizar el denominador para verificar si podemos cancelar los términos, el denominador tiene la forma $(a^{2} - b^{2})$.

	\item Factorizar denominador
	      $$\frac{x(x - 4)}{x^{2} - 16} = \frac{x(x - 4)}{(x + 4)(x - 4)}$$

	      Ya que en el numerador y en el denominador solo se está haciendo multiplicación; además de que tenemos términos semejantes en ambos lados, podemos cancelar los términos.

	\item Cancelar términos
	      $$\frac{x(x - 4)}{(x + 4)(x - 4)} = \frac{x\cancel{(x - 4)}}{(x + 4)\cancel{(x - 4)}} = \frac{x}{x + 4}$$
\end{enumerate}

\end{document}
